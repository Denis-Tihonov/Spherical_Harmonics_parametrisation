\documentclass[12pt,twoside]{article}
\usepackage{jmlda}

%\NOREVIEWERNOTES
\title
    [Аппроксимация фазовой траектории] 
    % Краткое название; не нужно, если полное название влезает в~колонтитул
    {Аппроксимация фазовых траектории квазипериодических сигналов на p-мерной единичной сфере с помощью сферических гармоник}
%\author
 %   [Автор~И.\,О.] % список авторов для колонтитула; не нужен, если основной список влезает в колонтитул
    %{Автор~И.\,О., Соавтор~И.\,О., Фамилия~И.\,О.} % основной список авторов, выводимый в оглавление
    %[Автор~И.\,О.$^1$, Соавтор~И.\,О.$^2$, Фамилия~И.\,О.$^2$] % список авторов, выводимый в заголовок; не нужен, если он не отличается от основного
    %\thanks
    %{Работа выполнена при финансовой поддержке РФФИ, проект \No\,00-00-00000.
   %Научный руководитель:  Стрижов~В.\,В.
   %Задачу поставил:  Эксперт~И.\,О.
   % Консультант:  Консультант~И.\,О.}
%\email
   % {author@site.ru}
%\organization
   % {$^1$Организация; $^2$Организация}
   
\abstract
    {	\textbf{Аннотация}: Решается задача аппроксимации фазовой траектории, построенной по квазипериодическому временному ряду.
    Фазовая траектория представлена в сферических координатах (и декартовых)  в виде проекции на единичную сферу в пространстве оптимальной размерности.
    Оптимальное пространство - это пространство минимальной размерности, в котором фазовая траектория не имеет ярковыроженных самопересечений на поверхности единичной сферы.
    Аппроксимация полученной фазовой траектории с помощью сферических гармоник.
    Эксперимент проведен на показателях акселерометра во время ходьбы.  

\bigskip
\textbf{Ключевые слова}: \emph {временные ряды; траекторное подпространство; фазовая траектория; сферические функции}.}

    
\begin{document}
\newcommand{\nsymbol}[2]{\medskip\hangindent=\parindent\hangafter=1\noindent $#1$ --- #2\par}
\newcommand{\nsymbolp}[3]{\nsymbol{#1}{#2 \dotfill\pageref{#3}}}

\maketitle

\section{Введение}
	Ставится задача построения модели аппроксимации квазипериодического временного ряда. Примерами таких сигналов являются показания акселерометра во время ходьбы и бега. 
	
	Для этого строится пространство фазовой тракеториии по выбранному временному ряду.  Это делается с помощью построения траекторной матрицы или матрицы Ганкеля. 
	
	Однако размерность траекторного пространства может оказаться избыточна. Это может приводить к неустойчивости исследуемых моделей и сложному описанию временного ряда. Для понижения размерности фазового пространства предлагается для сравнения использовать  следующие методы: метода главной компоненты (PCA)[?], метод сферической регрессии (Directional regression)[?] и метод вложений различной дисперсии (Distinguishing variance embedding)[?].
	
	В выбранном пространстве уменьшенной размерности предлагается спроецировать имеющуюся траекторию на $p$-мерную единичную сферу и перейти в $p-1$-мерное сферическое пространство. Полученную определенную на поверхности сферы функцию предлагается представить в виде ряда разложенного по сферическим функциям.
	
	


\section{Постановка задачи}
		
	%$\mathbf{X}$~--- матрица плана,
	%$$
	%\mathbf{X}=[\mathbf{x}^{\mathsf{T}}_1,...,\mathbf{x}^{\mathsf{T}}_m]^{\mathsf{T}}
	%$$
	%\mathbf{X}
	%\nsymbol
	%	{\mathbf{X}}
	%	{$\mathbf{X}=[x_j^i]\in\mathbb{R}^{m\times n}$,  $\mathbf{X}=[\x_1\T,\ldots,\x_m\T]\T$}
\end{document}
