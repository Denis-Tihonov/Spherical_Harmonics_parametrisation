\documentclass[12pt,twoside]{article}
\usepackage{jmlda}

%\NOREVIEWERNOTES
\title
    [Аппроксимация фазовой траектории] 
    % Краткое название; не нужно, если полное название влезает в~колонтитул
    {Аппроксимация фазовых траектории квазипериодических сигналов с помощью сферических гармоник}
%\author
 %   [Автор~И.\,О.] % список авторов для колонтитула; не нужен, если основной список влезает в колонтитул
    %{Автор~И.\,О., Соавтор~И.\,О., Фамилия~И.\,О.} % основной список авторов, выводимый в оглавление
    %[Автор~И.\,О.$^1$, Соавтор~И.\,О.$^2$, Фамилия~И.\,О.$^2$] % список авторов, выводимый в заголовок; не нужен, если он не отличается от основного
    %\thanks
    %{Работа выполнена при финансовой поддержке РФФИ, проект \No\,00-00-00000.
   %Научный руководитель:  Стрижов~В.\,В.
   %Задачу поставил:  Эксперт~И.\,О.
   % Консультант:  Консультант~И.\,О.}
%\email
   % {author@site.ru}
%\organization
   % {$^1$Организация; $^2$Организация}
\thanks
	{ }
\abstract
    {	\textbf{Аннотация}: Цель данной работы - построить модель аппроксимации наименьшей структурной сложности. Для этогов решается задача аппроксимации фазовой траектории, построенной по квазипериодическому временному ряду.
    Фазовая траектория представлена в сферических координатах (и декартовых)  в виде проекции на единичную сферу в пространстве оптимальной размерности.
    Оптимальное пространство - это пространство минимальной размерности, в котором фазовая траектория не имеет ярковыроженных самопересечений на поверхности единичной сферы.
    Аппроксимация полученной фазовой траектории с помощью сферических гармоник.
    ? Эксперимент проведен на показателях акселерометра во время ходьбы.  

\bigskip
\textbf{Ключевые слова}: \emph {временные ряды; траекторное подпространство; фазовая траектория; сферические функции}.}

    
\begin{document}
\newcommand{\nsymbol}[2]{\medskip\hangindent=\parindent\hangafter=1\noindent $#1$ --- #2\par}
\newcommand{\nsymbolp}[3]{\nsymbol{#1}{#2 \dotfill\pageref{#3}}}

\newcommand{\hookuparrow}{\mathrel{\rotatebox[origin=t]{270}{$\hookleftarrow$}}}
\newcommand{\hookdownarrow}{\mathrel{\rotatebox[origin=t]{90}{$\hookleftarrow$}}}

\maketitle

\section{Введение}
	Ставится задача построения модели аппроксимации квазипериодического временного ряда. Примерами таких сигналов являются показания акселерометра во время ходьбы и бега. 
	
	Для этого строится пространство фазовой тракеториии по выбранному временному ряду.  Это делается с помощью построения траекторной матрицы или матрицы Ганкеля. 
	
	Размерность траекторного пространства может оказаться избыточна. Это может приводить к неустойчивости исследуемых моделей и сложному описанию временного ряда. Для понижения размерности фазового пространства предлагается для сравнения использовать  следующие методы: метода главной компоненты (PCA)[?], метод сферической регрессии (Directional regression)[?] и метод вложений различной дисперсии (Distinguishing variance embedding)[?].
	
	В выбранном пространстве уменьшенной размерности предлагается спроецировать имеющуюся траекторию на $p$-мерную единичную сферу и перейти в $p-1$-мерное сферическое пространство. Полученную определенную на поверхности сферы функцию предлагается представить в виде ряда разложенного по сферическим функциям.
	
	


\section{Постановка задачи}
	По имеющемуся временному ряду $\mathbf{x}=[x_1,...,x_N]^{\mathsf{T}}$ строится траекторная матрица или матрица Ганкеля
	
		$$
		\mathbf{H_{x}} = 
		\begin{bmatrix} 
                  	x_{1} & x_{2} &\ldots&x_{n-1} &x_{n}\\
                  	x_{2} & x_{3} &\ldots&x_{n} &x_{n+1}\\
                  	\vdots& \vdots &\ddots&\vdots&\vdots\\
                  	x_{N-n+1} & x_{N-n+2} &\ldots&x_{N-1} &x_{N}\\
                   \end{bmatrix},
                   \eqno (1)
                   $$
                   
	где $N$-длинна временного ряда, $n$-ширина окна, не меньшая, чем предполагаемый период. Обозначим $t$-ую строку матрицы Ганкеля $\mathbf{H_{x}}$ за $\mathbf{x_{t}}$. Матрица $\mathbf{H_{x}}$ пребразуется к:
	
	
		$$
		\mathbf{H_{x}} = 
		\begin{bmatrix} 
                  	\mathbf{x_{1}}\\
                  	\mathbf{x_{2}}\\
                  	\vdots\\
                  	\mathbf{x_{m}}\\
                   \end{bmatrix},
                   \mathbf{x_t}=[x_{t},x_{t+1},\ldots,x_{t+n-1}] ,
                   m = N-n+1
                   \eqno (2)
                   $$
         Все векторы $\mathbf{x_{t}}$ принадлежат $\mathbb{H}_{\mathbf{x}} \subseteq \mathbb{R}^{n}$. Предполагается, что размерность траекторного пространства избыточна, поэтому предлагается исследовать некоторые проекции на траекторное подпространство.
Однако заранее неизвестно, в каком пространстве необходимо уменьшать размерность, поэтому задача приобретает следующий состоящий из двух вариантов вид:
		$$
		t \mapsto 
		\mathbf{x} \mapsto 
		\mathbb{H}_{x}^{n} \xrightarrow{}
		\mathbb{H}_{x}^{p} \xrightarrow{}
		\mathbb{S}_x^{(p-1)} \hookrightarrow
		[0,2\pi] \xrightarrow{f}
		r
		\eqno(3)
		$$
		$$
		t \mapsto 
		\mathbf{x} \mapsto 
		\mathbb{H}_{x}^{n} \xrightarrow{}
		\mathbb{S}_{x}^{n} \xrightarrow{}
		\mathbb{S}_x^{(p-1)} \hookrightarrow
		[0,2\pi] \xrightarrow{f}
		r
		\eqno(4)
		$$
	При понижении пространства, во-первых, требуется отыскать подходящий способ снижения размерности (линейные, нелинейные, нейросетевые методы), во-вторых, необходимо определить в каком пространсве сокращение размерности приведет к наименьшей потери информации и далее найти оптимальной сложности приближение при отысканию вложений. 
	\begin{Def}
	ываы
	\end{Def}
	$f:(\mathbf{w},\mathbf{x})\mapsto y$
	$\xrightarrow{1)}$
	$ \hookrightarrow{2)}$
	$dim\mathbb{A}^n$
	
	
\end{document}
